% Chapter Template

\chapter{Introduction and Motivation 1-2p} % Main chapter title

\label{Chapter1} % For referencing the chapter elsewhere, use \ref{Chapter1} 


\section{Introduction}

The Influence of the Internet of Things (IoT) in everyday life has been rising for years.
Connected devices are expected to number 20 billion \parencite{gartner0} by 2020 in nearly every industry.

According to Verizon \parencite{verizon} the use on digital devices in the media and entertainment industry increased by 120 \% in 2016 compared to 2013.
The industry was third in terms of accepting IoT,
with manufacturing (204\%) and finance and insurance (138\%) industries topping the chart.

Within the entertainment industry, escape rooms have been a growing sector since the first escape room launched 2007 in Japan.
Escape rooms generally follow the same structure: People are locked into a room, have to solve riddles
and get out in a defined period of time or will be released by a supervisor who watches the process to support
and assist in case of an emergency.

This field offers lots of possibilities for technical development, be it the use of different sensors, the use of Virtual Reality or flexible story-telling (depending on the users actions).
In this thesis, my goal was to create a suitable architecture and framework for further technical improvement for an existing escape room.

The faculty provided an escape room with microcontrolled riddles.
The former architect who designed the riddles and the set-up of the architecture of the room left the project.
Others have since tried to work with the existing architecture but struggled with it for reasons mentioned in Chapter \ref{Chapter3}.
The room supported the existing riddles but modifications were inconvenient to integrate.

This thesis will focus on developing an easy integration system for new riddles from different devices.
Furthermore, a user interface which supports communication with existing and new riddles dynamically was developed.
The second chapter will provide an overview about the research on topics relevant for this project. 
The third chapter will analyze the escape rooms architecture concerning the research. 
The fourth chapter will explain in further detail how the project was implemented. 
Chapter five will evaluate the implementation and examine future possibilites for the project.

%soll 2-3 seiten lang werden, vielleicht motivation
\section{Motivation}