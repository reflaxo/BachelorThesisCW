% Chapter Template
\chapter{Project Overview5-10p} % Main chapter title
%What is the problem and its effects?
\label{Chapter3} % For referencing the chapter elsewhere, use \ref{Chapter1} 
\section{Analysis}

\subsection{Layer Analysis}
%----------------------------------------------------------------------------------------
%	SECTION 1
%----------------------------------------------------------------------------------------
Refering to \ref{fig:gartnerIoT}, we analyzed the existing architecture of the escape room.

\begin{description}
	\item[Device Layer]\hfill \\
	      The Device Layer consisted of multiple Arduinos, and riddle-depending sensors and actuators for each riddle.
	      The gateway device service was an Adafruit Feather 32u4 which is connected via USB to a computer.
	\item[Communication Layer]\hfill \\
	      The communication within the room was set-up with RFM69HCW transceiver modules.\parencite{radiorange}
          The RFM69HCW is very cheap, easy to use and to set-up transceiver. They do packetization, error correction and auto-retransmit which makes them easy to handle.
	      They are designed for point-to-multipoint communication with one transceiver set as a gateway node which sends data to the other transceivers in the room.
	      There are two open-source libaries for the RFM69HCW, the LowPowerlab \parencite{LowPowerLab} and Radiohead \parencite{radiohead} libary.
	      The escape room used the LowPowerlab libary since the architect who designed the room prior to our adjustments was familiar with the libary.
          Now-a-days the Radiohead libary is the recommended libary \parencite{adafruitRecommends} since it's documented more thorough by an active community, kept up-to-date and is cross-platform friendly.
         
          The gateway transceiver was connected via USB to a computer where the data was forwarded via serialport communication (UART).
          In difference to SPI, UART is asynchronous and needs to be made synchronous to be interpreted by an other device. 
          Therefore, a stop- and start-bit is added to any message and transmission speed needs to be set on both sides.
          If the transmission speed differ, the messages can't be interpreted correctly by either side.
            
          Any node (riddle) within the room was recognized by a different nodeID and the gateway detected with a specified "GatewayID". 
        For security-reasons, password-encryption was used between the nodes. 

          The Arduinos within the room connected to the RFM69HCW via Serial Pheriphical Interface (SPI).
          SPI is a serial communication protocol common for micro-processor connection. 
          It uses an extra "Clock" (CLK) line to keep both sides in sync. 
          Only one side generates the clock signal (usually called the "master"). The other side is called the "slave".
          There can be multiple slaves, but only one master.
          The clock is an oscillating signal that tells 
          the receiver when to sample the bits on the data line exactly. 
          Bits are send either on the "high to low" or "low to high" edge of a CLK signal.
          If the master wants to send data to a slave, it's send via a MOSI line, for “Master Out / Slave In”.
          If a slave wants to send data to the master the data will be put on a third line called MISO for "Master In/Slave Out".
          The master will continue to generate a prearranged number of clock cycles, before the message is read by the master.
          As there is a MISO and a MOSI line, full-duplex communication where data is simultaniously send and received is possible with SPI.
          The fourth line needed to enable SPI communication is the "Slave Select"(SS) line which opens the communication channel. 
          If there is only one slave, the "SS" line is kept low (its active state) as long is the device is on. 

          \begin{figure*}[h]
            \centering
            \copyrightbox[r]{\includegraphics[width=75mm,scale=0.75]{Figures/adafruitSPI}}{\textcopyright https://learn.sparkfun.com/tutorials/serial-peripheral-interface-spi/all }
            \caption{Visualization of SPI communication}
          \end{figure*}
          
	\item[Information Layer]\hfill \\
        The Information Layer was build around a simple event model: the transceivers would send codes in a "Number/Number/Number..."-structure. 
        A node could be identified through the string it sent, since the first number would be the node's adressing number. 
        The other numbers would represent the current state of the Arduino with a "Switch-Case" scenario \ref{fig:switchCase}. 
	\item[Function Layer]\hfill \\
        A C++-Server would broadcast all incoming messages to any TCP-Client.    
        The server would forward any messages coming from a TCP-Client back to the serialport just as well.
	\item[Process Layer]\hfill \\
	If the Code matched with the Tcp-Clients list (in our case, in a Unity application), a Video in Unity would be played and Unity would send a "reset" code to the matching riddle. 
    Both lists were defined statically.] 
	Apart form the main functionality, the C++-Server offered a communication window where serialmessages 
    could be seen and sent manually. \ref{fig:c++window}.

\end{description}

    \begin{sidewaysfigure}
        \centering
        \includegraphics[width=200mm,scale=2]{Figures/escapeUmlOld}
        \caption{The old escape room architecture}
        \label{fig:oldEscapeUml}
    \end{sidewaysfigure}

\ref{fig:oldEscapeUml} provides further insight into the architecture

\begin{figure}[th]
    \centering
    \includegraphics[width=75mm,scale=0.75]{Figures/switchCase}
    \decoRule
    \caption[messages]{Example of the messages defined in the Arduino}
    \label{fig:switchCase}
\end{figure}

\begin{figure}[th]
    %\includegraphics[width=75mm,scale=0.75]{Figures/c++window}
    \decoRule
    \caption[messages]{Original Serial Window}
    \label{fig:c++window}
\end{figure}

\subsection{Workload analysis}
%SECTION TWOOO
In 2009, Kim Goodwin stated that 
„Interactive products and services tend to require four different types of work from users: 
cognitive, visual, memory, and physical“ \parencite{designDigitalAge} 
While an IoT-System is not always interactive, the goal of this thesis was very interaction-orientated: 
The goal was to help engineers expand the existing room which would require 
interaction with the Device-Layer (if they wanted to add hardware-functionality) 
and the Process-Layer (if they wanted to add software-functionality).
Therefore it makes sense to analyze this specific project with those aspects in mind. 

It should be mentioned that the room was well designed for the target user, which is an escape room customer, and 
the four areas of cognitive, visual, memory and physical work involved would already be pretty low.
Because this analysis will focus on an engineers point of view, the mentioned user in this case is the engineer who wants to extend or modify the room in some way.

\subsubsection{Cognitive Work}
Engineering products usually demands some kind of cognitive work. Still, there are hurdles that can be avoided,
like finding out wether to click "yes" "no" or "cancel" in a confirmation dialog can be made clearer if the question is phrased easily.
In this case, the cognitive work needed to add a riddle or a functionality was very high:

Device Layer:
\begin{enumerate}
    \item The engineer needs to understand the transport protocol and therefore needs to 
    \begin{enumerate}
        \item Set-Up the RFM69HCW with an Arduino or an Adafruit Feather, which requires drivers and another libary
        \item Understand the communication within the LowPowerlab-libary
        \item Look up the other riddles codes to avoid using a %belegte nodeID
    \end{enumerate}   
    \item Understand Arduino-coding 
    \item Understand working with the "Switch-Case"-scenario used for communication 
\end{enumerate}  

Process Layer:
\begin{enumerate}
    \item Understand C++ to change the Server (f.e. to communicate with an upper-level protocol)
    \item Understand C\# and Low-Level-Socket communication to make changes in Unity 
    and gain an overview about the communication since it's in separate files (one File per Video and one for the Tcp Socket) 
\end{enumerate}  

\subsubsection{Visual Work}
Visual work means how much the user needs to search for features in a product visually.
The visual work within the architecture was low, as the interface to work with was simple.
It didn't offer needed functions for the user, like buttons for often used features(reset, "send feedback"...).

\subsubsection{Memory Work}
Memory work is measured in how much a user has to remember to succeed in a task. 
Typical examples are passwords, command names, and file names.
In its former state, the architecture demanded a high amout of memory work from a developer:
\begin{enumerate}
    \item The developer has to translate "Number/Number/Number..." codes whenever he wants to understand the serialport-messages 
    \item The developer has to remember a different code when he wants to send an order to the device
    \item Anyone who starts the room has to remember to first start the c++ server and afterwards the unity application on the desktop, or connection will fail.
    \item The developer has to enable the "enable feedback" mode for each node manually if he wants to see all messages send
    \item The developer has to enable another mode if he wants to interact with the riddles, and the riddle won't react hardware-wise in that mode.
\end{enumerate} 

\subsubsection{Physical Work}
IoT-Projects always involve some kind of physical work for a developer: the developers need to switch between hardware and software to test the devices for example.
The escape room fits into that scenario but doesn't make testing physically harder than it needs to be in most aspects. 
One aspect that makes changes harder is seemingly unavoidable - most of the hardware is hidden, therefore mostly difficult to access. 
Since an escape room is made for customers who expect the illusion of in this case, a spaceship, 
touch-sensitive hardware would impact that illusion.
